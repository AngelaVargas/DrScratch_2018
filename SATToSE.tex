\documentclass[a4paper,twocolumn,10pt]{article}

\newif\ifdraft
\drafttrue
%\draftfalse

\usepackage[utf8]{inputenc}
\usepackage[english]{babel}
\usepackage{graphicx}
\usepackage{multirow}
\usepackage{tabularx}
\usepackage{array}
\usepackage{geometry}
\usepackage{cuted, ragged2e}
\usepackage{lipsum, array, booktabs, caption}
\usepackage{multirow}
\usepackage{graphicx}
\usepackage{xspace}
\usepackage{color}
\usepackage[usenames,dvipsnames]{xcolor}
\usepackage{hyperref}


\input{macros}

%opening
\title{A Preliminary Analysis of Scratch Projects Using a Publicly Available Dataset}

\author{
Ángela Vargas-Alba \\ Universidad Rey Juan Carlos\\
                Madrid, Spain \\ a.vargasa@alumnos.urjc.es
\and
Gregorio Robles \\Universidad Rey Juan Carlos \\
                Fuenlabrada, Madrid, Spain \\grex@gsyc.urjc.es   
}

\date{}


\begin{document}
\maketitle


\small \textbf{\textit{Abstract}—One of the most widely used tools for the initiation in the world of programming is
Scratch, a visual programming language oriented to education. Its use has become more popular in the recent years with millions of
projects shared and that can be viewed (and reused) by other Scratch users. 
In this paper, we perform a preliminary analysis of a public available dataset of 250K Scratch projects obtained from the Scratch repository, with the objective of extract correlations between different data.
The goal of this research is to gain knowledge on the Scratch ecosystem in order to have better tools and methods for learners and educators.}

\begin{center}
\section*\normalsize{I. INTRODUCTION}\\
\end{center}

At present time, an increasingly tendency of introducing the programming in the
classrooms is developing, as part of the movement that tries to boost the development of Computational Thinking (CT)
skills among younger students~\cite{wing2006computational}. A large number of learners has decided to use Scratch as
a learning language (and platform)~\cite{resnick2009scratch}. Currently, Scratch
has in its web platform more than 30 millions of shared projects and more than 20 millions of registered users.
Thanks to this high activity, it is possible to inspect different aspects which relate projects each other.

Thus, the motivation of this paper is, on one hand, to analyze which characteristics can relate different projects
and, on the other hand, how metadata of the project is related itself. This analysis could be useful for some tools or
applications in the future.

In order to accomplish this analysis, we use the dataset offered by the MSR Data Paper ``A Dataset of Scratch Programs: Scraped, Shaped and
Scored dataset'' by Aivaloglou et al.~\cite{aivaloglou2017dataset}. For this dataset, authors collect data from the web
interface of the Scratch project repository, by means of a scraping program. It offers a JSON file for each of the listed
projects and, subsequently, they obtained the JSON files for 250,163 projects. They parsed this dataset and obtained
a list of Scratch projects with metadata and program data. Finally, they use the Dr.Scratch assessment tool~\cite{moreno2015dr} 
to evaluate them.

In this paper, we have utilized this dataset, which contained more than 250K projects from the Scratch repository. The
main aim is to analyze the relationship between the projects, and extract useful conclusions about it. 
This analysis can help educators in the assessment of Scratch projects. In addition, it can facilitate the learning process
in different aspects, for example, getting users to do more social projects or avoiding that users tend to repeat
code or blocks from other projects.


The structure of the work has been divided according to the following way: In section~\ref{sec:construction} the process
used to retrieve the dataset is presented. Section~\ref{sec:methodology} includes the methodology used to obtain the analyzed
correlations. The results are presented in Section~\ref{sec:results} and, finally, in the Section~\ref{sec:conclusions}, we
present the conclusions reached.



\section{Construction of the dataset}
\label{sec:construction}

\subsection{Data collection}

To collect the data from the web interface of the Scratch
project repository, we used the dataset created by the 2017 MSR Data paper~\cite{aivaloglou2017dataset}.
This dataset was composed as from two files: \textit{metadata.csv} and
\textit{code.csv}. These files were formed by heterogeneous data, that is,
they contained different types of data. For this reason, we initially
did a data filtering. With a script programmed on \textit{python}, we read
both csv files and removed \textit{nul} or ``strange'' values from them.
We understood ``strange'' values as those values that contained some special
characters.

Respect to the file \textit{code.csv}, it was a file with large dimensions. 
For this reason, we utilized only the 10\% of its content in order to work in a more
comfortable way. This new file, \textit{code\_10\%\_headers.csv},
contained information about only 20,061 of the total projects. For each
project, it contained different data, which will be described in
subsection~\ref{subsec:description}, so that it was formed by 3,988,988 total lines.

Based on the filtering process, we removed 3,166 lines, but the number of
different projects was the same, 20,061. We removed some data related with
projects, but the same number of projects was maintained.

On the other hand, we removed the characteristics related to the
parameters of each project. We considered that these data was not relevant
for the analysis. Finally we obtained the file \textit{code\_10\%\_filt.csv}.

Respect to the other file, \textit{metadata.csv}, it was no necessary to carry
out the filtering process. Data of this file was homogeneous, so it was
possible to work directly. However, it contained metadata which we didn't
consider useful in the analysis of this paper. For this reason, we reduced
this file and removed some columns.

\subsection{Data description}
\label{subsec:description}

Scratch is a visual programming language whose projects are based on a structure
composed of different blocks. The code of a Scratch program, is formed by scripts,
which are defined as a set of code blocks. Each script belongs to a sprite, an object
with its own associated code. In Figure~\ref{tab:scratch}, is shown an example
Scratch script which belongs to a given sprite.

\begin{figure}
\begin{center}
\includegraphics[height=4cm]{images/scratch.png}
\caption{Example of a Scratch program with a script in the same sprite.}
\label{tab:scratch}
\end{center}
\end{figure}

In this paper, we are going to work with two files:
\textit{metadata.csv} and \textit{code\_10\%\_filt.csv}. 
The data related with each file is outlined in Table~\ref{tab:schema}.

\begin{table*}
  \centering\renewcommand{\arraystretch}{1.2}
   \begin{tabular}{@{}lccc@{}}
    \toprule
    \multicolumn{1}{c}{\textbf{File}} & \textbf{Key} &\textbf{Attribute(Description)} \\
    \midrule
    \multirow{8}{*}{\textit{code\_10\%\_filt.csv}} & ProjectID & Scratch project Id \\
                & Script-rank & Project-level ranking of script \\
                & Sprite-type & Type of sprite the script is in\\
                & Sprite-name & Name of sprite the script is in\\
                & Coordinates & X-Y location of script in Scratch editor\\
                & Total-blocks & Number of blocks comprising the script \\
                & Line & Script-level ranking of block \\
                & Block-type & Type of block the script is in\\
    \hline
    \multirow{8}{*}{\textit{metadata.csv}} & p\_ID & Scratch project Id \\
                & project-name & Name given to project \\
                & username & Author’s Scratch username\\
                & total-views & Project views number \\
                & total-remixes & Project remixes number \\
                & total-favorites & Total users favoriting \\
                & total-loves & Users ‘loving’ the project \\
                & Mastery & Dr. Scratch total mastery score\\
    \bottomrule
    \end{tabular}
    \caption{Database schema: Data structure and description}
    \label{tab:schema}
\end{table*}


In Table~\ref{tab:schema}, the last key \textit{Mastery}, is the result of Dr.Scratch
score. Dr.Scratch is a web platform of free software that allows users to analyze their projects 
carried out in Scratch, in a simple way~\cite{moreno2015dr}. With Dr. Scratch, both students and
educators can analyze if their projects have been programmed correctly, learn from their mistakes or
receive feedback to improve their code, thus developing their capacity of
Computational Thinking (CT).

In order to assign the punctuation of CT to a certain project, Dr.Scratch assesses the demonstrated
level by the programmer in the seven following aspects: abstraction and decomposition of problems, logical
thinking, synchronization, parallelism, algorithmic notions of flow control, 
interactivity with the user and representation of the information. The 
assessment of each one of these concepts can be 0, 1, 2 or 3 points, being
the projects, therefore, contained in the range from 0 to 21 point. According
to the final punctuation obtained, Dr. Scratch differentiates three levels of
projects: basics, from 0 to 7 points, developing from 7 to 15 points, and
professional, with punctuation between 15 a 21 points.

\section{Methodology}
\label{sec:methodology}

Once the dataset is filtered and ordered, we can analyze them.
The main purpose of the paper is to find correlations among Scratch
projects. In order to reach this objective, we have developed an analysis
process.

Initially, we focused on the characteristic ``total-blocks''. We generated a new
dataset grouped by ``projectID'', ``script-rank'' and ``total-blocks''. In this way,
we obtained the total number of blocks for each Scratch project and its project
Id. The following stage was to analyze among the total blocks of each project, 
which of them were different. We calculated the variety of block types. 
In order to obtain this new characteristic, we grouped the unique block types for
each Scratch project, obtaining a new dataset.

We related both dataset and obtained two correlations: the correlation between
``total-blocks'' and ``variety-blocks'', and the correlation between ``total-blocks''
and ``repeated-blocks'', represented in Figure~\ref{tab:corr_1}.


\begin{figure}
\includegraphics[height=4cm]{images/1.png}
\includegraphics[height=4cm]{images/2.png}
\caption{Correlations between total blocks and blocks type.}
\label{tab:corr_1}
\end{figure}

The following analysis was focused on correlate the total number of blocks per project
and its metadata. For this process, it was necessary relate both CSV files. For this
reason, the first stage was to select from \textit{metadata.csv}, only the id of the
projects contained in file \textit{code\_10\%\_filt.csv}.

Once that we obtained both files filtered, we analyzed the correlation between 
``total-blocks'' and ``total-views'', ``total-blocks'' and ``total-remixes'', 
``total-blocks'' and ``Mastery'' and ``variety-blocks'' and ``Mastery''. The result
of these correlations is showed in Figure~\ref{tab:corr_2}.

\begin{figure}
\begin{center}
\includegraphics[height=4cm]{images/3.png}
\includegraphics[height=4cm]{images/4.png}
\includegraphics[height=4cm]{images/5.png}
\includegraphics[height=4cm]{images/6.png}
\caption{Correlations between total blocks and metadata.}
\label{tab:corr_2}
\end{center}
\end{figure}

Finally, we considered relevant for the analysis of the paper, to correlate some
characteristics of metadata. In this way, we analyzed the correlation between: 
``Mastery'' and ``total-views'', ``total-remixes'' and ``total-views'',
``total-views'' and ``total-favorites'' and ``total-views'' and ``total-loves''.
The results are represented in Figure~\ref{tab:corr_3}.

\begin{figure}
\begin{center}
\includegraphics[height=4cm]{images/7.png}
\includegraphics[height=4cm]{images/8.png}
\includegraphics[height=4cm]{images/9.png}
\includegraphics[height=4cm]{images/10.png}
\caption{Correlations between metadata.}
\label{tab:corr_3}
\end{center}
\end{figure}


\section{Results}
\label{sec:results}

In the first stage of analysis, we obtained interesting results, as is shown in
Figure~\ref{tab:corr_1}. The value of first correlation is 0.53, which indicates that the 
relation between the number of total blocks in a project is not proportional
to its variety, analyzed as different block types. On the contrary, the
correlation obtained between total blocks in a project and repeated blocks is
0.99. This result indicates that those projects which have a greater number of 
blocks, are formed by more quantity of repeated blocks.

In the second analysis, shown in Figure~\ref{tab:corr_2}, we can observe that 
a clear correlation between data does not exist. These results indicate that projects which have
a greater number of blocks not necessarily receive a greater number of views and
remixes by other users. On the other hand, projects which have a greater number of
blocks, neither obtain a greater final mastery score.

Finally, contrary to the previous analysis, Figure~\ref{tab:corr_3} shows a high correlations 
between metadata of a Scratch project. The correlation obtained between the total
views of a given project and the total number of remixes, is 0.86. This result 
indicates that those projects that have realized a greater quantity of remixes,
are more socials and receive a greater quantity of views. If we analyze the 
correlation between total views of a project and its total number of loves and
favorites, the correlation is even greater. The results obtained are 0.94 and 0.95,
respectively. Projects that have received a greater number of views, will receive more
favorites and loves from other users. However, the correlation between total views and the
final mastery of a project, is almost null. Projects with more views, not necessarily
will have a greater mastery.

Table~\ref{tab:values} shows a summary of the obtained results.

\begin{table}
   \begin{tabular}{lr}
    \toprule
    \textbf{Attributes} &\textbf{Corr.} \\
    \midrule
	Total-blocks, Variety-blocks & 0.53\\
        Total-blocks, Repeated-blocks & 0.99\\
        Total-blocks, Total-views & -0.01\\
        Total-blocks, Total-remixes & -0.01\\
        Total-blocks, Mastery & -0.01\\
        Variety-blocks, Mastery & -0.02 \\
        Mastery, Total-views & 0.03\\
        Total-remixes, Total-views & 0.86\\
        Total-views, Total-loves & 0.94\\
        Total-views, Total-favorites & 0.95\\
    \bottomrule
    \end{tabular}
    \caption{Summary of obtained correlations}
    \label{tab:values}
\end{table}


\section{Conclusions}
\label{sec:conclusions}

We have presented an analysis of data obtained from Scratch repository, thanks
to the previous work of the paper~\cite{aivaloglou2017dataset}, which includes the source code of the Scratch
projects, their metadata, and their programming mastery scoring results.
After a filtering process, we have obtained the dataset analyzed in this paper. We
have searched different correlations between code source of a Scratch project and metadata.
The analysis performed can facilitate the development of some applications or tools
in education area and computing assessment~\cite{robles2018ontools}.


\bibliographystyle{plain} 
\bibliography{samplebib}

\end{document}
