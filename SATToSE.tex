\documentclass[a4paper,twocolumn,10pt]{article}
\usepackage[utf8]{inputenc}
\usepackage[english]{babel}
\usepackage{graphicx}
\usepackage{flushend}
\usepackage{titlesec}

%opening
\title{Título}
\author{Autor}
\date{}


%\renewcommand{\thesection}{\normalsize\Roman{section}.}

\begin{document}
\maketitle


\small \textbf{\textit{Abstract}—One of the most widely used tool for the initiation in the world of programming is
Scratch, a visual programming language oriented to education. Its use has intensified in the recent years with millions of
projects that can be shared and viewed by other Scratch users. 
In this paper, we analyze a dataset of 250K Scratch projects obtained from the Scratch repository,  with the objective of
extract correlations between different data.}

\begin{center}
\section*\normalsize{I. INTRODUCTION}\\
\end{center}

Presently, increasingly the tendency of introducing the programming in the
classrooms is developing. Nevertheless, there is a lack of formation in this matter and Scratch is a great
usefulness in order to supply this deficiency, due to the fact that, beside helping the educator in the
assessment tasks, it supposes a stimulus to encourage the learners to continue improving their
programming skills. With this tool it is expected to guide the collective in their first steps giving
support, both educators and learners, in the programming sector. \par
In recent years, increasingly the users decide to utilize Scratch as learn programming method. Currently, Scratch
has in its web platform more than 30 millions of shared projects and more than 20 millions of registered users.\par
Thanks to the high activity in the web platform of Scratch, it is possible to inspect different aspects which relate projects
each other. Thus, the motivation of this paper is, on one hand, to analyze which characteristics can relate different projects
and, on the other hand, how metadata of the proyect is related itself. \par
In order to accomplish this analysis, we have based on the paper ``A Dataset of Scratch Programs: Scraped, Shaped and
Scored dataset'' [1]. In this work, authors collect data from the web interface of the Scratch project repository, by means 
of a scraping program. It retrieves a JSON file for each of the listed projects and, subsequently, they obtained the JSON files for
250,163 projects. Finally, they parsed this dataset and obtained a list of Scratch projects with metadata and program data. \par
In this paper, we have utilized this dataset, which contained more than 250K projects from the Scratch repository, with the
main aim of analyze if there is any kind of relationship between them, and extract useful conclusions about it.\par
The structure of the work has been divided according to the following way: In section II. Dataset construction, it will be
developed the process of obtaining the dataset. Section III includes the methodology used to obtain the analyzed correlations.
The results are presented in Section IV and, finally, in the Section V, we present the conclusions reached.

\begin{center}
\section*\normalsize{II. DATASET CONSTRUCTION}
\end{center}

\subsection*\normalsize{\textit{A. Data collection}}\\

To collect the data from the web interface of the Scratch
project repository, we utilized the dataset created by the paper [1].
This dataset was composed as from two files: \textit{metadata.csv} and
\textit{code.csv}. These files were formed by heterogeneous data, that is,
they contained different types of data. For this reason, we initially
did a data filtering. With a script programmed on \textit{python}, we read
both csv files and remove \textit{nul} or ``strange'' values from them.
We understood ``strange'' values as those values that contained some special
characters.\par
Respect to the file \textit{code.csv}, it was a file with large dimensions,
so that we utilized only the 10\% of its content in order to work in a more
comfortable way. This new file, \textit{code\_10\%\_headers.csv},
contained information about only 20061 of the total projects. For each
project, it contained different data, which will be described in
section B, so that it was formed by 3988988 total lines. \par
Based on the filtering process, we removed 3166 lines, but the number of
different projects was the same, 20061, so we removed some data related with
projects, but the same number of projects was maintained.\par
On the other hand, we decided to remove the characteristics related to the
parameters of each project. We considered that these data was not relevant
for the analysis, so finally we obtained a file
\textit{code\_10\%\_filt.csv}.\par
Respect to the other file, \textit{metadata.csv}, it was no necessary to carry
out the filtering process. Data of this file was homogeneous, so it was
possible to work directly. However, it contained metadata which we didn't
consider useful in the analysis of this paper. For this reason, we reduce
this file and removed some columns.

\subsection*\normalsize{\textit{B. Data description}}\\

The two files with which we are going to work in this paper are:
\textit{metadata.csv} and \textit{code\_10\%\_filt.csv}. 
The data related with each file is outlined in Table I.

\begin{center}
\begin{tabular}{ l | c | r  }
   \textbf{File} & \textbf{Key} & \textbf{Attribute(Description)} \\
 \hline  
   \textit{code\_10\%\_filt.csv} & ProjectID & Scratch project Id \\
    & Script-rank & Project-level ranking of script \\
    & Sprite-type & Type of sprite the script is in\\
    & Sprite-name & Name of sprite the script is in\\
    & Coordinates & X-Y location of script in Scratch editor\\
    & Total-blocks & Number of blocks comprising the script \\
    & Line & Script-level ranking of block \\
    & Block-type & Type of block the script is in\\
    \textit{metadata.csv} & p\_ID & Scratch project Id \\
    \hline
    & project-name & Name given to project \\
    & username & Author’s Scratch username\\
    & total-views & Project views number \\
    & total-remixes & Project remixes number \\
    & total-favorites & Total users favoriting \\
    & total-loves & Users ‘loving’ the project \\
    & Mastery & Dr. Scratch total mastery score\\
    \hline
    
\end{tabular}
Table I. DATABASE SCHEMA: DATA STRUCTURE AND DESCRIPTION
\end{center}

\par In Table I, the last Key \textit{Mastery}, is the result of Dr. Scratch
score. Dr. Scratch is a web platform of free software or open source that 
allows users to analyze their projects carried out in Scratch, in a simple
way. With Dr. Scratch, both students and educators can analyze if their
projects have been programmed correctly, learn from their mistakes or
receive feedback to improve their code, thus developing their capacity of
Computational Thinking (CT). \par
In order to assign the punctuation of CT to a certain project, Dr. Scratch
is based on assess the demonstrated level by the programmer in the seven
following aspects: abstraction and decomposition of problems, logical
thinking, synchronization, parallelism, algorithmic notions offlow control, 
interactivity with the user and representation of the information. The 
assessment of each one of these concepts can be 0, 1, 2 or 3 points, being
the projects, therefore, contained in the range from 0 to 21 point. According
to the final punctuation obtained, Dr. Scratch differentiates three levels of
projects: basics, from 0 to 7 points, developing from 7 to 15 points, and
professional, with punctuation between 15 a 21 points.


\section*\normalsize{III. METHODOLOGY}

\section*\normalsize{IV. RESULTS}

\section*\normalsize{V. CONCLUSIONS}


\end{document}
