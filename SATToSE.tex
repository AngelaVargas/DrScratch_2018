\documentclass[a4paper,twocolumn,10pt]{article}
\usepackage[utf8]{inputenc}
\usepackage[english]{babel}
\usepackage{graphicx}
\usepackage{flushend}

%opening
\title{Título}
\author{Autor}
\date{}

\setcounter{secnumdepth}{0} 
\setcounter{tocdepth}{1} 

\newcounter{ns}
\addtocounter{ns}{1} 

\begin{document}
\maketitle


\small \textbf{\textit{Abstract}—One of the most widely used tool for the initiation in the world of programming is
Scratch, a visual programming language oriented to education. Its use has intensified in the recent years with millions of
projects that can be shared and viewed by other Scratch users. 
In this paper, we analyze a dataset of 250K Scratch projects obtained from a repository,  with the objective of extract
correlations between different data.}

\section\normalsize{I. INTRODUCTION}

Presently, increasingly the tendency of introducing the programming in the
classrooms is developing. Nevertheless, there is a lack of formation in this matter and Scratch is a great
usefulness in order to supply this deficiency, due to the fact that, beside helping the educator in the
assessment tasks, it supposes a stimulus to encourage the learners to continue improving their
programming skills. With this tool it is expected to guide the collective in their first steps giving
support, both educators and learners, in the programming sector. \par
In recent years, increasingly the users decide to utilize Scratch as learn programming method. Currently, Scratch
has in its web platform more than 30 millions of shared projects and more than 20 millions of registered users.\par
In this paper, we have selected randomly 250K projects from the Scratch repository with the main aim of analyze if there
is any kind of relationship between them, and extract useful conclusions about it. In section II. Dataset construction, 
it will be developed the process of obtaining the dataset. Section III includes the methodology used to obtain
the analyzed correlations. The results are presented in Section IV and, finally, in the Section V, we present
the conclusions reached.


\section\normalsize{II. DATASET CONSTRUCTION}

\section\normalsize{III. METHODOLOGY}

\section\normalsize{IV. RESULTS}

\section\normalsize{V. CONCLUSIONS}


\end{document}
